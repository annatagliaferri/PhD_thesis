% English abstract
\chapter*{Abstract}
\markboth{\MakeUppercase{Abstract}}{\MakeUppercase{Abstract}}

Every year, foodborne illnesses affect approximately 10\% of the population, resulting in the death of \num{420000} people, therefore effective preventative measures should be put in place to avoid significant consequences on society. A promising solution is provided by sensors and biosensors. These advanced devices convert the interaction between a target analyte and a recognition element into a detectable signal (optical, electrical, etc.) via a transducer. The transducer employed in this project employs is the electrolyte-gated field-effect transistor (EG-FET), an innovative three-electrode transducer that is intrinsically able to amplify small electric field variations at the gate electrode, leading to the generation of larger output current variations. The advantages of using an electrolyte are manifold: electrolytes maintain biological molecules in a quasi-native environment, preventing degradation (crucial for DNA, protein, or whole-cell detection) and also contribute to signal amplification with their high dielectric constants, which result in high capacitance in the double layer. A higher capacitance causes a stronger modulation of the channel conductance for a given gate voltage, thereby improving the sensitivity of the device. Another important component of the EG-FET is the semiconductor, where the charges flow to generate the response current of the platform. This project employs semiconducting single-walled carbon nanotubes (SWCNTs), chosen for their unique electronic, chemical and mechanical properties, such as high mobility and flexibility. Despite having numerous advantages, SWCNTs are susceptible to oxygen and water molecule, which can affect their conductivity and mobility upon interaction, compromising platform performance. This project demonstrates a simple, inexpensive method to enhance EG-CNTFET stability and reliability: a lipophilic membrane is dropcasted onto the channel, protecting the SWCNTs from interactions with molecules in the environment. This results in faster stabilization, occurring in 33 minutes, significantly less than other advanced devices reported in the literature, which require double that time. Furthermore, the devices were able to maintain stability when stored in air, as opposed to previous devices that needed a protective nitrogen-filled environment, and they were resistant to electric stress for up to 12 hours. Additional testing proved that these devices could also withstand five repetitive rounds of experiments with very little variability among the different cycles. 

This platform functionality is still partially unknown, requiring further investigation to fully understand its underlying mechanisms. Consequently, COMSOL simulations were conducted, yielding preliminary results; however, further work is required for complete device characterization and conclusive findings. 

Once the platform was stable, it was tested as the transducer for ammonium and histamine sensors. The ammonium sensor, functionalized with an ion-sensitive membrane on the gate, detected ammonium ions linearly from \SI{0.01}{mM} to \SI{100}{mM}. The histamine sensor, functionalized with aptamers on the gate, requires further optimization to enhance its sensitivity; however, preliminary testing indicates detection capabilities down to \SI{0.01}{\micro M} of histamine.

