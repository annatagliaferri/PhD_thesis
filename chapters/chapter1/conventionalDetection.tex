The detection of foodborne hazards is traditionally based on microbiological, biochemical, and analytical chemistry techniques; these include culture-based approaches, immunology-based methods, PCR-based techniques, chromatography and mass spectrometry \citep{velusamyOverview2010,iammarinoAdvanced2022}. The purpose of this section is to provide a brief explanation of how these technologies work and to describe their advantages and disadvantages.

\subsection{Microbiological and Biochemical Techniques}
\label{sec:microbiological_detection}

\paragraph{Culture-Based Methods}
Culture-based methods are the oldest techniques used to test the presence of foodborne pathogens in food samples. They usually consist of sequential steps of enrichment, selective and differential plating, confirmation and strain characterization \citep{kabirazConventional2023}.
These methods are highly reliable and allow for the isolation of live organisms for further characterization. On the other hand, cultures are time-consuming, requiring up to 72 hours or more to obtain measurable results \citep{saravananMethods2021}. They are also labor-intensive, requiring skilled personnel, and may underestimate the number of pathogens in samples in the case of presence of viable but non-culturable (VBNC) bacteria \citep{liImportance2014}.

\paragraph{Immunology-Based Methods}
Immunology-based methods refer to those assays that employ antibodies to identify the desired analytes; such techniques include enzyme-linked immunosorbent assay (ELISA), immunochromatography assay (ICA, also known as lateral flow assay), and immunomagnetic separation (IMS) \citep{gaoResearch2024}. In ELISA, either the antibody or the antigen is attached to a plate and the binding occurs upon recognition \citep{engvallEnzymelinked1971}. In ICA, a sample flows along a paper strip onto which antibodies are adsorbed; upon antibody-antigen interaction a coloured line becomes visible indicating the presence of the analyte \citep{zengImmunochromatographic2021}.
IMS uses magnetic beads coated with antibodies to capture specific analytes from samples; after cycles of magnetic separation and washing, the analyte will be isolated and ready to be characterized \citep{moyanoMagnetic2019, gagicFully2020}.
These methods benefit from the high specificity of the interaction between antibody and analyte, and are capable of detecting not just live microorganisms but also toxins and spores. However, they often require either animal use in the form of a hybridoma or prior genetic knowledge for antibody production in bacterial cultures, and are susceptible to interference from contaminants in the sample \citep{velusamyOverview2010}.

\paragraph{PCR-Based Methods}
PCR-based methods, which include the most common conventional PCR, real-time PCR, and reverse transcriptase PCR, amplify the target DNA sequences present in a sample in order to detect down to very low concentrations of analyte \citep{salazarPolymerase2015}. First, nucleic acid is extracted from the sample, then the target sequences are amplified thanks to specific primers that bind to them. In conventional PCR, amplified DNA is visualized by means of gel electrophoresis; real-time PCR makes use of fluorescent dyes or probes that allow to monitor the amplification process as it happens, making it a quantitative technique; reverse-transcriptase PCR uses a reverse transcriptase to convert RNA into its complementary DNA (cDNA) strand before amplifying it \citep{liLuminescent2020}. More PCR-based methods have been developed over the years, for more clarity and examples, refer to \citet{singhCritical2014}. PCR-based methods are highly sensitive and can provide results within a few hours, making them much faster than culture-based techniques. However, their drawbacks include their inability to distinguish between live and dead cells, which can lead to false positives and an overestimation of DNA concentration. Moreover, PCR techniques are expensive and require specialized equipment and technical expertise, limiting their accessibility in field settings \citep{velusamyOverview2010,kabirazConventional2023}.

\subsection{Analytical chemistry techniques}
\label{sec:chemistry_detection}

\paragraph{Chromatography}
Chromatography is a term used to refer to a set of analytical chemistry techniques that separate the components of a sample based on their interactions with a stationary phase (the column) and a mobile phase (the eluent) \citep{malikLiquid2010,zhangRapid2021}. Gas chromatography (GC) is most effective for volatile and semi-volatile compounds, for example pesticides \citep{walorczykPesticide2016, nolvachaiMultidimensional2017, fengGas2019}; on the other hand, liquid chromatography (LC) is mostly used non-volatile analytes like mycotoxins, bacterial toxins and phycotoxins \citep{yamatodaniHighperformance1985,nieHighperformance2019,fedorenkoRecent2023,quintanilla-villanuevaRecent2024}.

While chromatography is able to provide accurate indication of the presence foodborne hazards, sample preparation can be expensive, complex and time-consuming, making chromatography not always the best choice for this task \citep{vuckovicCurrent2012,heDetection2021}.


\paragraph{Mass Spectrometry}
The objective of mass spectrometry (MS) is to generate ions from the compounds in the sample, which will subsequently reach the mass analyzer separately according to their mass-to-charge ratio \citep{franzenMassenspektrometrie1969}. For this reason, MS is selective, specific and highly sensitive: indeed, MS is able to detect nanograms of analytes, also in complex matrices \citep{nwachukwuRecent2024}. To improve its performance, MS is often used in tandem with chromatographic techniques (\ie{} GC-MS or LC-MS), allowing to identify and quantify foodborne hazards \citep{putriApplication2022,zhongUntargeted2022}. However, like chromatography, MS-based methods require expensive equipment, and often need extensive data analysis. Moreover, MS is a destructive technique, thus the sample gets lost in the process and cannot be retrieved for further characterization \citep{cortes-herreraLiquid2019,artaviaSelected2021}.

\paragraph{Spectroscopy-Based Techniques}
Spectroscopic methods, which include infrared (IR) spectroscopy, Raman spectroscopy, and nuclear magnetic resonance (NMR) spectroscopy, are other prominent methods for the detection of foodborne hazards \citep{heDetection2021}. These techniques exploit the interaction of electromagnetic radiation with matter to investigate molecular structures and chemical compositions. Namely, IR and Raman spectroscopy can identify specific chemical bonds and functional groups depending on the molecular bond vibrations \citep{quApplications2015,petersenApplication2021}.
NMR spectroscopy exploits the magnetic moment of atomic nuclei which interact with electromagnetic radiation, emitting different signals based on the chemical environment that surrounds the nuclei themselves \citep{hatzakisNuclear2019}. While these methods are non-destructive and often require minimal sample preparation, their sensitivity is usually lower than that of other techniques, thus limiting their practical application \citep{emwasStrengths2015}.

\subsection{Limitations of traditional detection methods}
\label{sec:limitations_traditional}

Despite their widespread use, traditional detection methods for foodborne hazards hide many weaknesses and cannot always meet the demands of food safety monitoring; indeed, they rely on time-consuming protocols, skilled labor, and expensive, usually bulky, equipment that cannot be utilized for real-time, on-site applications.
These limitations highlight the need for innovative approaches. In this context, sensors and biosensors have emerged as promising tools for food safety, as described in the next section.
