
\chapter{Introduction}
\label{cap:introduction}
\thispagestyle{empty}

\newpage
\thispagestyle{empty}
\ % Empty page
\newpage

\section{Motivation}

Foodborne diseases, despite their significant impact, often receive less attention than other areas of medical research, such as cancer, neurodegenerative diseases, and rare conditions. While the numbers associated with foodborne illnesses are quite alarming, they frequently fail to capture public interest beyond isolated incidents like the recent tragic death of a young child in Italy due to Escherichia coli-contaminated raw milk cheese, which caused hemolytic-uremic syndrome. Regulatory agencies, including the U.S. Food and Drug Administration, the European Food Safety Authority, and the Centers for Disease Control and Prevention, implement stringent food safety measures and rigorous testing. However, these measures can only guarantee food wholesomeness up to a certain point, primarily regulating production and distribution but with limited control once food enters our homes. Furthermore, even production and distribution points cannot be monitored constantly. Current pathogen and toxin detection systems, typically accessed by production and distribution centers and used primarily for occasional checks or post-intoxication investigations, highlight the need for more accessible and widely distributed systems. The motivation for this PhD project stems from the desire to develop simple, cost-effective tools that can be used by everyone to check for pathogens or food toxins, mirroring the increasing accessibility of diagnostics in the medical field, such as the inexpensive, rapid, and user-friendly lateral flow sensors used for COVID-19 diagnostics and pregnancy tests.

\section{Thesis aim}

The objective of this thesis is to develop a stable, sensitive sensor for foodborne hazards. This sensor aims to address the critical need for rapid and reliable detection of contaminants in the food supply, ultimately contributing to improved food safety and public health. Current methods for detecting foodborne pathogens and toxins, such as traditional culturing techniques and ELISA assays, can be time-consuming, labor-intensive, and may lack the sensitivity required to detect low levels of contamination. For example, culturing Salmonella can take up to 72 hours, delaying crucial intervention measures. Enzyme-linked immunosorbent assays (ELISAs), while faster, can sometimes produce false negatives or require specialized equipment. Therefore, a sensor offering faster, more sensitive, and potentially portable detection capabilities would represent a significant advancement.

The stability aspect of the sensor refers to its ability to maintain its performance characteristics (sensitivity, selectivity, and accuracy) over extended periods and under varying environmental conditions, such as temperature and humidity fluctuations, which are common in food processing and storage environments. For instance, a sensor that degrades quickly or is easily affected by temperature changes would be impractical for real-world applications. The sensitivity of the sensor is crucial for detecting low concentrations of foodborne hazards, enabling early detection and prevention of outbreaks. This is particularly important for toxins like aflatoxin, where even trace amounts can pose a significant health risk.

Indeed, this thesis proves that a sensor can be fabricated using an electrolyte-gated field effect transistor (EG-FET) that has been stabilized by using a lipophilic membrane on the carbon-nanotube-based channel. EG-FETs are particularly attractive for biosensing due to their label-free detection capabilities and potential for miniaturization. The lipophilic membrane, typically composed of polymers like polyvinyl chloride (PVC) with plasticizers such as dioctyl sebacate, serves to improve the long-term stability of the sensor in complex biological media. To prove this, the gate of the device was functionalized with two different recognition elements: first an ion-selective membrane to detect ammonium ion, then aptamers for histamine. The ammonium ion-selective membrane typically incorporates ionophores such as nonactin, which selectively binds ammonium ions, creating a potential difference at the membrane-electrolyte interface that modulates the transistor's current. Histamine, on the other hand, was detected using aptamers, which are single-stranded DNA or RNA molecules that bind to specific target molecules with high affinity. For example, a specific aptamer sequence known to bind histamine with a dissociation constant in the nanomolar range could be used. While these are not the worst toxins that can contaminate food, such as botulinum toxin or aflatoxins, they prove that our platform can work with different analytes, demonstrating the versatility of the sensor design. The successful detection of both a small ion (ammonium) and a larger organic molecule (histamine) highlights the adaptability of the EGFET platform for a wide range of sensing applications, from environmental monitoring to food safety analysis.

\section{Scientific collaborations}

The work is highly multidisciplinary, requiring expertise in electronic engineering, physics, chemistry, and biology. Full device fabrication and characterization demand diverse skills and specialized equipment. Therefore, we collaborated with Dr. Stefano Bonaldo and Prof. Alessandro Paccagnella from the Department of Information Engineering at the University of Padova. This collaboration provided additional information regarding the protective membrane and helped to select optimal characterization experiments that we could independently conduct. Furthermore, a collaboration was established with the Theory and Modelling group at Linköping University, led by Prof. Igor Zozoulenko, to develop a mathematical model describing operational mechanism of the EG-FET and to conduct simulations for a deeper understanding of its functionality.

\section{Thesis outline}
 
The thesis comprises five chapters. Chapter \ref{cap:chapter1} introduces the background, discussing foodborne illnesses and their detection methods, providing context for the research. It then highlights the limitations of current detection technologies, such as their speed or sensitivity, and proposes a solution using innovative sensors and biosensors. Finally, it includes a section on relevant mathematical models and simulations used to support the experimental work. Chapter \ref{cap:chapter2} details the specific materials and methods employed in the experiments to achieve the results presented in subsequent chapters. Chapter \ref{cap:chapter3} presents the work undertaken to accelerate analysis and enhance the stability of the developed devices, crucial for practical application. Chapter \ref{cap:chapter4} focuses on the mathematical model describing the EG-FET's operation, including preliminary simulation results that require further optimization to match experimental findings. Lastly, Chapter \ref{cap:chapter5} demonstrates the developed platform's functionality as a signal transducer for both ammonium and histamine sensors, showcasing its versatility.

\newpage
\thispagestyle{empty}
\ % Empty page
\newpage