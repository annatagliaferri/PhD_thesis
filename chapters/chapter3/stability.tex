Carbon nanotubes are a particularly attractive semiconductor material because of their electrical, mechano-chemical and chemical properties that allow them to have high mobility and thermal conductivity and flexibility \citep{joshiUnderstanding2018}; this has enabled their use in numerous electronics applications. Nevertheless, they have limitations that need to be addressed. Indeed, the electronic properties of CNTs are dependent on their atomic structures and on charge transfer and chemical doping effects by various molecules \citep{daiCarbon2002}. An example of doping is that of oxygen: the molecule is normally adsorbed to the CNTs acting as a hole dopant, thus conferring p-type behaviour \citep{daiCarbon2002,mceuenSinglewalled2002}. It is for this reason that a first source of instability could be found in the changes in oxygen concentration in the environment, which could alter the doping levels and consequently the electrical characteristics \citep{mceuenSinglewalled2002}. Humidity and water are further causes of instability, as water molecules can adsorb on the CNTs, leading to hysteresis in their electrical behavior and affecting the threshold voltage \citep{mceuenSinglewalled2002,kimHysteresis2003,zaumseilSemiconducting2019}. Finally, structural defects and vacancies, along with mechanical deformations, such as bending or strain in flexible electronics, can potentially influence CNT electrical properties thus having  detrimental effect on CNT overall stability \citep{daiCarbon2002,rodriguez-manzoCreation2009}.
It is important to note that random CNT network that have been solution-processed add another source of instability: indeed the charge transport can be limited by hopping or tunneling between nanotubes, being influenced by residual surfactant and overlapping nanotubes. These phenomena, not only influence stability, but also have an effect on device-to-device reproducibility \citep{zaumseilSemiconducting2019}.

Several strategies have been developed to counteract CNT instability, including passivation layers \citep{avourisMolecular2002,molazemhosseiniRapidly2021}, improved contact engineering \citep{avourisMolecular2002,javeyCarbon2004,naderiReview2016}, controlled doping methods \citep{avourisMolecular2002,naderiReview2016}, and careful management of the CNT network structure \citep{zaumseilSemiconducting2019,shkodraElectrolytegated2021}.

With the considerations described herein, this chapter is devoted to the research and study of an easy and inexpensive method to improve the stability and reproducibility of EG-CNTFETs.
First, it was found that \citet{joshiUsing2018} developed a lipophilic membrane that improved the performance of EG-CNTFETs. The improvement was evaluated in the reduction of the gate leakage current by two orders of magnitude, the twofold increase of the \ratio{} and the decrease of the hysteresis by about tenfold.
The primary drawback of using this membrane lies in the decrease of the drain-source current after membrane encapsulation; this occurs because the membrane creates a thin barrier that reduces the effective potential drop across the channel and there is increased resistance between source and drain due to additional scattering centers.

Further evidence of the benefits that a lipophilic membrane has on the stability and performance of EG-CNTFETs comes from experiments conducted previously in our laboratory. Specifically, it was seen that using an ion-sensitive membrane not only improved the parameters describing the performance of the device, but also improved the current over time, making long measurements more reliable \citep{petrelliFlexible2022,petrelliNovel2022,petrelliMethod2023}

Therefore, from these observations, a device was developed on whose channel a lipophilic membrane was deposited as a passivation layer and its performance was evaluated. Three different EG-FET geometries have been tested with the aim to find the one stabilizing the fastest and the best performing. The first geometry, presented in Section \ref{sec:big_channel} followed a well-established design; Section \ref{sec:small_distance} reports the results obtained for a slightly modified structure, where the channel has been moved closer to the gate, aiming to enhance electrostatic control and thus improve stability; the last geometry, depicted in Section \ref{sec:small_channel} maintained the proportions of the first structure, but with channel and gate areas four times smaller, with the hypothesis that reducing the CNT area would lead to faster stabilization.