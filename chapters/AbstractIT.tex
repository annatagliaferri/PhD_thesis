% Italian abstract
\chapter*{Sommario}
\markboth{\MakeUppercase{Sommario}}{\MakeUppercase{Sommario}}

Ogni anno, le malattie trasmesse dagli alimenti colpiscono circa il 10\% della popolazione, causando la morte di \num{420000} persone, pertanto è necessario adottare misure preventive efficaci per evitare conseguenze significative sulla società. Una soluzione promettente è fornita da sensori e biosensori. Questi dispositivi avanzati convertono l'interazione tra un analita target e un elemento di riconoscimento in un segnale rilevabile (ottico, elettrico, ecc.) tramite un trasduttore. Il trasduttore impiegato in questo progetto è il transistor a effetto di campo elettrolitico (EG-FET), un innovativo trasduttore a tre elettrodi intrinsecamente in grado di amplificare piccole variazioni di campo elettrico all'elettrodo di gate, portando alla generazione di variazioni di corrente di uscita maggiori. I vantaggi dell'utilizzo di un elettrolita sono molteplici: gli elettroliti mantengono le molecole biologiche in un ambiente quasi nativo, prevenendone la degradazione (fondamentale per il rilevamento di DNA, proteine o cellule intere) e contribuiscono anche all'amplificazione del segnale con le loro elevate costanti dielettriche, che determinano un'elevata capacità nel doppio strato. Una capacità più elevata determina una modulazione più forte della conduttanza del canale per una data tensione di gate, migliorando così la sensibilità del dispositivo. Un altro componente importante dell'EG-FET è il semiconduttore, dove le cariche fluiscono per generare la corrente di risposta della piattaforma. Questo progetto impiega nanotubi di carbonio a parete singola semiconduttori (SWCNT), scelti per le loro proprietà elettroniche, chimiche e meccaniche uniche, come elevata mobilità e flessibilità. Nonostante i numerosi vantaggi, gli SWCNT sono sensibili all'ossigeno e alle molecole d'acqua, che possono influenzare la loro conduttività e mobilità in caso di interazione, compromettendo le prestazioni della piattaforma. Questo progetto dimostra un metodo semplice ed economico per migliorare la stabilità e l'affidabilità dell'EG-CNTFET: una membrana lipofila viene dropcasting sul canale, proteggendo gli SWCNT dalle interazioni con le molecole nell'ambiente. Ciò si traduce in una stabilizzazione più rapida, che avviene in 33 minuti, significativamente meno rispetto ad altri dispositivi avanzati riportati in letteratura, che richiedono il doppio di quel tempo. Inoltre, i dispositivi sono stati in grado di mantenere la stabilità quando conservati all'aria, a differenza dei dispositivi precedenti che necessitavano di un ambiente protettivo riempito di azoto, ed erano resistenti allo stress elettrico fino a 12 ore. Ulteriori test hanno dimostrato che questi dispositivi potevano anche resistere a cinque cicli ripetitivi di esperimenti con pochissima variabilità tra i diversi cicli.

Questa funzionalità della piattaforma è ancora parzialmente sconosciuta, richiedendo ulteriori indagini per comprenderne appieno i meccanismi sottostanti. Di conseguenza, sono state condotte simulazioni COMSOL, che hanno prodotto risultati preliminari; tuttavia, sono necessari ulteriori lavori per una caratterizzazione completa del dispositivo e risultati conclusivi.

Una volta che la piattaforma era stabile, è stata testata come trasduttore per sensori di ammonio e istamina. Il sensore di ammonio, funzionalizzato con una membrana sensibile agli ioni sul gate, ha rilevato gli ioni di ammonio in modo lineare da \SI{0.01}{mM} a \SI{100}{mM}. Il sensore di istamina, funzionalizzato con aptameri sul gate, richiede un'ulteriore ottimizzazione per migliorarne la sensibilità; tuttavia, test preliminari indicano capacità di rilevamento fino a \SI{0.01}{\micro M} di istamina.