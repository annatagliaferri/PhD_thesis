\chapter{Summary and future perspectives}
\label{Conclusions}
\thispagestyle{empty}

\newpage
\thispagestyle{empty}
\ % Empty page
\newpage

\section{Summary}

The conclusion of this paper is that the EG-CNTFET platform was effectively improved by enhancing its stability, achieving a stabilization time twice as fast as state-of-the-art devices reported in the literature. Specifically, this study demonstrates a rapid and cost-effective method to enhance the stability of SWCNT-based EG-FETs by employing a dropcasted lipophilic membrane. This membrane significantly decreases the stabilization time compared to the state-of-the-art, halving it to approximately 34 minutes. Notably, after stabilization, the current displays a linear increase of \SI{11}{\nA/min} over time, with a coefficient of determination of 99\%. Furthermore, performance evaluation revealed a 151\% improvement in the ON/OFF ratio over 1 hour, accompanied by a decrease in the absolute value of \vth{} during the same period. Long-term stability was confirmed through continuous transfer characteristic measurements over \SI{4}{\hour} and \SI{12}{\hour}, which also demonstrated a linear trend of \ids{} over time, with increases of \SI{1}{\nA/min} and \SI{0.7}{\nA/min}, respectively. The devices also demonstrated resilience to electrical stress, withstanding repeated data acquisition cycles. A single device successfully underwent the stabilization protocol up to 5 times with excellent reproducibility after \SI{30}{\min}, exhibiting less than 2\% variability at the conclusion of the measurements. The functionality of this device was further investigated through the development of a mathematical model and COMSOL simulations, yielding a set of equations that define the working mechanism and describe the underlying physics. When functionalized with an ion-selective membrane sensitive to ammonium and with aptamers to detect histamine, the EG-CNTFETs were successfully employed as biosensors. The ammonium sensors displayed a linear range between \SIrange{0.01}{100}{mM} with a coefficient of determination of 0.94 and a sensitivity of \SI{0.143}{\uA} per decade. For histamine detection, the devices exhibited a linear range between \SIrange{0.01}{100}{\micro M} with a coefficient of determination of 0.96 and a sensitivity of \SI{0.028}{\uA} per decade. In conclusion, the lipophilic membrane protecting the SWCNTs offers substantial advantages, reducing stabilization time and improving device performance over time, without compromising its intended use.

\section{Future perspectives}

Further investigation and optimization are warranted. The resistance imposed by the membrane on the channel requires further study. Furthermore, the mechanisms by which ions become trapped within the membrane, and the influence of these ions on device performance, should be investigated. Structured studies should be performed to optimize the working voltage range of the devices, maximizing the signal while avoiding water electrolysis and device damage. Simulation efforts also require further development; current simulations do not perfectly align with experimental results, necessitating continued study in this area. Upon completion of these simulations, a more thorough understanding of the working mechanisms will be achieved, facilitating easier optimization of experimental parameters. Finally, improvements to device sensitivity and the execution of selectivity tests are necessary.

\newpage
\thispagestyle{empty}
\ % Empty page
\newpage