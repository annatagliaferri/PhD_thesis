% German abstract
\chapter*{Abstrakt}
\markboth{\MakeUppercase{Abstrakt}}{\MakeUppercase{Abstrakt}}

Jedes Jahr erkranken etwa 10\% der Bevölkerung an lebensmittelbedingten Krankheiten, die mehr als \num{420000} Todesfälle verursachen. Wirksame Präventionsmaßnahmen sind daher notwendig, um schwerwiegende gesellschaftliche Folgen zu vermeiden. Sensoren und Biosensoren bieten hier einen vielversprechenden Lösungsansatz. Diese hochentwickelten Geräte wandeln die Wechselwirkung zwischen einem Analyten und einem Erkennungselement über einen Wandler in ein detektierbares Signal (optisch, elektrisch etc.) um. Der in diesem Projekt verwendete Wandler ist ein elektrolytgesteuerter Feldeffekttransistor (EG-FET), ein innovativer Dreielektrodenwandler, der in der Lage ist, kleine elektrische Feldänderungen an der Gateelektrode zu verstärken und dadurch größere Stromänderungen zu erzeugen. Die Verwendung eines Elektrolyten hat mehrere Vorteile: Elektrolyte halten die biologischen Moleküle in einer quasi-nativen Umgebung und verhindern so deren Abbau (entscheidend für den Nachweis von DNA, Proteinen oder ganzen Zellen) und tragen durch ihre hohe Dielektrizitätskonstante, die zu einer hohen Kapazität in der Doppelschicht führt, zur Signalverstärkung bei. Eine höhere Kapazität führt zu einer stärkeren Modulation der Kanalleitfähigkeit bei gegebener Gatespannung und verbessert damit die Empfindlichkeit des Bauelements. Eine weitere wichtige Komponente des EG-FET ist der Halbleiter, durch den die Ladungen fließen, um den Antwortstrom der Plattform zu erzeugen. In diesem Projekt werden halbleitende einwandige Kohlenstoffnanoröhren (Single-Walled Carbon Nanotubes, SWCNTs) verwendet, die aufgrund ihrer einzigartigen elektronischen, chemischen und mechanischen Eigenschaften wie hohe Mobilität und Flexibilität ausgewählt wurden. Trotz ihrer zahlreichen Vorteile sind SWCNTs anfällig gegenüber Sauerstoff- und Wassermolekülen, die bei Wechselwirkung ihre Leitfähigkeit und Mobilität und damit die Leistungsfähigkeit der Plattform beeinträchtigen können. Dieses Projekt demonstriert eine einfache und kostengünstige Methode zur Verbesserung der Stabilität und Zuverlässigkeit von EG-CNTFETs: Eine lipophile Membran wird auf den Kanal getropft und schützt die SWCNTs vor Wechselwirkungen mit Molekülen in der Umgebung. Dies führt zu einer schnelleren Stabilisierung, die innerhalb von 33 Minuten eintritt und damit deutlich kürzer ist als bei anderen in der Literatur beschriebenen Geräten, die die doppelte Zeit benötigen. Darüber hinaus blieben die Geräte auch stabil, wenn sie an der Luft gelagert wurden, im Gegensatz zu früheren Geräten, die eine schützende Stickstoffumgebung benötigten. Auch gegen elektrischen Stress waren sie bis zu 12 Stunden beständig. Weitere Tests zeigten, dass die Geräte auch fünf wiederholte Testzyklen mit sehr geringer Variabilität zwischen den Zyklen überstanden.

Die Funktionalität dieser Plattform ist noch teilweise unbekannt und bedarf weiterer Untersuchungen, um die zugrunde liegenden Mechanismen vollständig zu verstehen. Daher wurden COMSOL-Simulationen durchgeführt, die vorläufige Ergebnisse lieferten. Für eine vollständige Charakterisierung des Bauelements und abschließende Ergebnisse sind jedoch weitere Arbeiten erforderlich.

Nachdem die Plattform stabil war, wurde sie als Wandler für Ammonium- und Histaminsensoren getestet. Der Ammoniumsensor, funktionalisiert mit einer ionensensitiven Membran am Gate, detektierte Ammoniumionen linear von \SI{0,01}{mM} bis \SI{100}{mM}. Der Histaminsensor, funktionalisiert mit Aptameren am Gate, bedarf weiterer Optimierung, um seine Sensitivität zu erhöhen. Vorläufige Tests deuten jedoch auf eine Nachweisfähigkeit von bis zu \SI{0,01}{\micro M} Histamin hin.