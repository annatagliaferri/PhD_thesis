\subsection{Materials}

The devices are composed of a polyimide substrate (\SI{50}{\um}), titanium, gold, and semiconducting single-walled carbon nanotubes (SWCNTs, \SI{95}{\%}, Merck). Additional chemicals employed during fabrication were
ma-n 1420 photoresist (Microresit), ma-D 533 S developer (Microresist), both used during photolithography (Section \ref{sec:lithography}), carboxymethyl cellulose (CMC, Merck), a surfactant necessary during SWCNT deposition, and nitric acid (\ce{HNO3}, Merck), to remove said surfactant at the appropriate time. The lipophilic membrane was synthesized combining \SI{5}{\%} tetradodecylammonium tetrakis(4-chlorophenyl) borate (Merck), \SI{32.3}{\%} poly(vinyl chloride) (PVC, Merck), and \SI{62.7}{\%} dioctyl sebacate (Merck); the reagents were then diluted in \SI{1}{\ml} tetrahydrofuran (THF, Merck). A second, ionophore-containing membrane used for functionalization purposes consisted of \SI{0.2}{\%} nonactin (Merck), \SI{30.8}{\%} PVC (Merck), and \SI{69}{\%} ortho-nitrophenyl octyl ether (O-NPOE, Merck); it was also diluted in \SI{1}{\ml} THF. Histamine aptamers (Microsynth AG, Balgach, Switzerland) in combination with thiolated polyethylene glycol (PEG, Merck) were used as a second functionalization strategy. Phosphate-buffered saline (PBS, Merck, pH 7.2--7.6, \SI{1}{tablet} per \SI{200}{\ml}) and ferricyanide/ferrocyanide (Fe[\ce{CN}]\textsuperscript{3-/4-}, Merck) were subsequently employed for electrical characterization of the fabricated devices and sensors. Other materials widely employed throughout the process included acetone (Merck), isopropyl alcohol (IPA, Merck).

\subsection{Equipment}

% Litho + evap
The instrumentation used for electrode fabrication comprised oven, spin coater, a mask aligner equipped with a UV lamp, and an evaporator. 
% CNT spray
Subsequently, the CNT solution was prepared using a horn sonicator (Fisherbrand FB-505), an ultrasound sonicator, and a centrifuge (Thermo Scientific SL 16 with 15-6 rotor); the solution was then deposited using a spray coater equipped with an industrial air atomizing spray valve (Nordson EFD, USA). A plasma asher (Diener Femto series) was employed prior to CNT deposition to ensure a clean surface and to increase contact angle.
% Functionalization
Functionalization first required cleaning through an ozone cleaner (Ossila), then aptamer preparation using Zeba spin desalting columns (\SI{7}{\kilo\dalton} MWCO, \SI{0.5}{\ml}, Thermo Fisher Scientific, Switzerland) and a centrifuge (Eppendorf 5420 with FA-24x2 rotor).
% Characterization
Device imaging and characterization was carried out after, employing techniques such as optical microscopy (Axio Imager, Carl Zeiss Microscopy GmbH, Germany), scanning electron microscopy (SEM; JSM-IT 100, JEOL, Japan), X-ray photoelectron spectroscopy (XPS; Omicron Nanotechnology Multiprobe UHV-surface-analysis system with Al~K\textalpha{} source, \SI{1486.7}{\electronvolt}), atomic force microscopy (AFM; Nanosurf CoreAFM, Switzerland) and profilometry (KLA Tencor P-6). Electrical characterization was later performed using a probe station connected to a Keysight B1500A semiconductor device parameter analyzer and a potentiostat (VersaSTAT 4, Princeton Applied Research, USA).

Simulations and numerical analyses were performed using COMSOL Multiphysics 6.2, executed on the Tetralith high-performance computing cluster managed by the National Academic Infrastructure for Supercomputing in Sweden (NAISS), more specifically by the National Supercomputer Centre (NSC)
at Linköping University. Tetralith is the primary system of NSC, with a cluster \num{1908} compute nodes, each with two Intel Xeon Gold 6130 CPUs (32 cores).

